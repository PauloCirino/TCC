% -----------------------------------------------------------------------------
%   Arquivo: ./01-elementos-pre-textuais/resumoPt.tex
% -----------------------------------------------------------------------------



\begin{resumo}

O  tema  desse trabalho  de  conclusão  de  curso  é  a criação de um software  para  automatizar etapas do processo de \textit{benchmarking}  de  modelos  de  aprendizado  de  máquina. 

Foi feito  um  software, escrito na linguagem de programação R, capaz  de  comparar  modelos  de  aprendizado  de máquina  disponíveis  na  literatura com  modelos  criados  pelo  usuário.  O  software  faz  o carregamento  das  bases  de  dados  padrões,  suas  chamadas, as métricas de avaliação e  os  testes  estatísticos de qualidade para os problemas de classificação e regressão.

Será discutido nesse trabalho, o impacto social de software livre e os aspectos técnicos e definições matemáticas do aprendizado supervisionado. Por fim será apresentado o pacote e um exemplo de como utiliza-lo na função de auxiliar o  \textit{benchmarking}  entre os modelos.
 
\textbf{Palavras-chave}: Software Livre. Aprendizado de Máquina. Linguagem R. Teste de Modelos.

 

\end{resumo}



% -----------------------------------------------------------------------------
%   Escolha de 3 a 6 palavras ou termos que descrevam bem o seu trabalho. As palavras-chaves são utilizadas para indexação.
%   A letra inicial de cada palavra deve estar em maiúsculas. As palavras-chave são separadas por ponto.
% -----------------------------------------------------------------------------
