% -----------------------------------------------------------------------------
%   Arquivo: ./02-elementos-textuais/trabalhosRelacionados.tex
% -----------------------------------------------------------------------------



\chapter{Estatísticas de Teste}
\label{chap:testStatistics}

\section{O que é Estatística de Teste}
Uma estatística de teste é o valor que é calculado a partir de dados durante um teste de hipóteses. O valor dessa estatística mede o grau de concordância entre uma amostra de dado e a hipótese nula, que por sua vez pode ser rejeitada ou não \cite{Casella2002}. 

Em um teste estatístico, as hipóteses são premissas a serem testadas. Tradicionalmente existem duas hipóteses, a nula e a alternativa, em que o objetivo do teste é conservadoramente rejeitar a hipótese nula à favor da hipótese alternativa. Dessa forma, os testes são feitos tal que o objetivo que deseja-se testar é descrito pela hipótese alternativa.

Os tipos de testes que serão implementados no projeto são os testes pareados paramétricos e não-paramétricos.

\section{Teste Pareado}
Os testes pareados são aqueles utilizados para comparar um conjuntos de medidas de duas populações e avaliar se elas são diferentes. Nessa situação, tradicionalmente a hipótese nula é a diferença entre estatística entre as populações, e a hipótese alternativa pode ser que uma é maior que a outra ou que são diferentes.

Uma vantagem de utilizar os testes pareados, é que eles tornam possível a comparação de múltiplas populações, fazendo vários testes 2 à 2.

\section{Testes paramétrico e não-paramétrico}


