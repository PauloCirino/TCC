% -----------------------------------------------------------------------------
%   Arquivo: ./02-elementos-textuais/trabalhosRelacionados.tex
% -----------------------------------------------------------------------------



\chapter{Estatísticas de teste}
\label{chap:testStatistics}

\section{Definição de estatística de teste}
Uma estatística de teste é o valor que é calculado a partir de dados durante um teste de hipóteses. O valor dessa estatística mede o grau de concordância entre uma amostra de dado e a hipótese nula, que por sua vez pode ser rejeitada ou não \cite{Casella2002}. 

Em um teste estatístico, as hipóteses são premissas a serem testadas. Tradicionalmente existem duas hipóteses, a nula e a alternativa, em que o objetivo do teste é conservadoramente rejeitar a hipótese nula à favor da hipótese alternativa. Dessa forma, os testes são feitos tal que o objetivo que deseja-se testar é descrito pela hipótese alternativa.

Os tipos de testes que serão implementados no projeto são os testes pareados paramétricos e não-paramétricos.

\section{Teste pareado}
Os testes pareados são aqueles utilizados para comparar um conjuntos de medidas de duas populações e avaliar se elas são diferentes. Nessa situação, tradicionalmente a hipótese nula é a diferença entre estatística entre as populações, e a hipótese alternativa pode ser que uma é maior que a outra ou que são diferentes.

Uma vantagem de utilizar os testes pareados, é que eles tornam possível a comparação de múltiplas populações, fazendo vários testes 2 à 2.

\section{Testes paramétrico e não-paramétrico}
Um teste estatístico paramétrico é aquele que faz suposições sobre os parâmetros da
distribuição geradora das populações que estão sendo testadas. Desta forma, um teste
não-paramétrico é aquele que implica a ausência dessas suposições, de forma mais direta é
aquele que não supões nada sobre os parâmetros da função de distribuição de probabilidade
geradora \cite{LowryFerro}.

Na maioria das situações práticas, os teste não-paramétricos são vantajosos quando as
populações de teste são muito pequenas, possuem estrutura ordinal, ou são melhor representadas pela mediana. Quase que em todas as demais situações, os testes paramétricos
se comportam de forma mais confiável e geram resultados com maior potência estatística \cite{Casella2002}.

Além disso, os teste paramétricos não estão limitados pela suposição de que a dispersão
amostral é igual nas populações de teste, diferente da maioria dos não-paramétricos, e
tampouco funciona apenas para funções geradoras normais. Dado um tamanho amostral
suficientemente grande, qualquer distribuição pode ser aplicado para um teste que assume
normalidade \cite{KernsFerro}.

