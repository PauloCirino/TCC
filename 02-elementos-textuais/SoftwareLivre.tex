% -----------------------------------------------------------------------------
%   Arquivo: ./02-elementos-textuais/trabalhosRelacionados.tex
% -----------------------------------------------------------------------------



\chapter{Software Livre}
\label{chap:SoftwareLivre}

\section{O Que é Software Livre}

A definição de software livre apresenta os critérios para determinar se um programa de software é qualificado como software livre. Essa definição pode mudar conforme o momento histórico, mas atualmente é definida pela GNU como \textit{software que os usuários têm a liberdade de executar, copiar, distribuir, estudar, alterar e melhorar} \cite{Inc2012} . 

Nesse contexto, o conceito de livre diz respeito à "liberdade de expressão", não como gratuito. Com essas liberdades, os usuários, tanto individualmente ou coletivamente, controlam o programa e o que ele pode fazer. Quando os usuários não controlam o programa, chamamos ele de programa \textit{não livre} ou \textit{proprietário}. 

Em linhas gerais, um software livre deve, obrigatoriamente, obedecer a quatro liberdades \cite{Inc2012, Williams, Lessig2002} :
\begin{compactenum}
	\item A liberdade de executar o programa como desejar, para qualquer propósito;
	\item A liberdade de estudar como funciona o programa e mudá-lo para que ele faça a sua computação como você deseja;
	\item A liberdade de redistribuir cópias para que você possa ajudar seu vizinho;
	\item A liberdade de distribuir cópias de suas versões modificadas para outros.
\end{compactenum}

Naturalmente, essas liberdades estão relacionadas às informações divulgadas entre desenvolvedor e usuário. A liberdade \textbf{2}, por exemplo, implica na necessidade de o desenvolvedor divulgar abertamente o código fonte de um software, e permitir que esse seja modificado. Além disso, é obvio que para o usuário terem a liberdade de decidir sobre sua computação, o software livre, por sua vez, não pode utilizar nenhum código \textit{não livre}.

É importante notar que não existem, nessas liberdades, quaisquer limitações relacionadas ao uso comercial do software. Isso significa que empresas podem cobrar pelo uso do software, em versões modificadas ou não.

Além das liberdades básicas de um software livre, existem abaixo delas, licenças de uso. Cada licença específica, restringe ou garante mais liberdades sobre o software, sem afetar as quatro fundamentais. De forma forma geral, existem quatro famílias de licenças : \textit{Permissiva},  \textit{Fracamente Protetiva},  \textit{Fortemente Protetiva} e \textit{Protetiva de Rede} \cite{Williams}. 

\section{História do Software Livre}
O surgimento do movimento de software livre está altamente atrelado a academia e ao desenvolvimento dos sistemas operacionais UNIX, GNU e Linux.

O UNIX têm suas origens na \textit{joint venture}, lançada no final da década de 1960 pela \textit{Bell Labs} e MIT para criar um novo sistema operacional chamado \textit{Multics}. Utilizando o conhecimento adquirido nesse projeto, alguns dos programadores desenvolveram paralelamente um sistema operacional para oferecer mais flexibilidade aos usuários, que nomearam UNIX. 

Em 1975, Ken Thompson juntamente com Bill Joy e Chuck Haley começaram a distribuir uma versão \textit{open source} do UNIX chamada BSD. No ano seguinte, o lançamento de uma edição revista foi denominada 2BSD.

Em 1984, o programador Richard Stallman fundou o Projeto GNU. A GNU GPL permitia aos usuários modificar o código e distribuir a versão melhorada sob a mesma licença. O sistema operacional GNU não tinha um \textit{kernel}, até que Linus Torvalds desenvolveu o \textit{kernel} do Linux. Em 1992, o \textit{kernel} do Linux foi integrado no sistema operacional GNU.

Nos anos seguintes surgiu a introdução de muitas versões comerciais e aprimoradas do sistema operacional Linux por fornecedores como Red Hat, Mandriva e Novell.

Com a criação de sistemas operacionais que poderiam ser utilizados com total liberdade, surgiu então uma demanda por softwares livres que funcionassem nesses sistemas. Da mesma forma que a comunidade se juntou para aprimorar a base do \textit{kernel} do Linux, eles juntaram e fizeram os mais diversos softwares para atender a demanda. 

Nos dias atuais existem diversas comunidades que fazem os mais variados tipos de softwares utilizando os mesmos princípios criados por Richard Stallman. Os softwares livres difundiram na sociedade, e hoje são peças fundamentais para infraestrutura computacional, periféricos, celulares e virtualmente qualquer outro dispositivo computacional.

\section{A Importância Social do Software Livre}
Atualmente, os softwares livres são fundamentais principalmente em três áreas sociais: à proteção da liberdade individual, ao avanço da computação e à acessibilidade da educação e conhecimento.

\subsection{Software Livre à Favor da Proteção da Liberdade Individual}
Uma celebre frase da comunidade de software livre diz, 'Os softwares não livres, onde o usuários não controlam o programa, o programa controla os usuários' \cite{Williams}. 

Essa frase, resume uma preocupação crescente com o software proprietário, a de que sempre há   alguma entidade, que controla o programa e através dele, exerce poder sobre seus usuários. 

Esse poder é enxergado por alguns como uma afronte ao direito de privacidade do indivíduo. Hoje, serviços de busca e redes sociais, utilizam dos dados de navegação para gerar propagandas sob-medidas.  Muitas pessoas, consideram a forma que essas empresas manipulam as informações como uma forma de censura e venda de informação confidencial. 

Em alguns casos, empresas que desenvolvem softwares proprietários foram ligadas a escândalos onde propositalmente construíram \textit{backdoors} em seus produtos que dão acesso a informações sem as devidas permissões dos usuários. Um exemplo é o caso do \textit{ Kindle}, que possui uma \textit{backdoor} que permite apagar livros \cite{GNUOperatingSystem} .

Um outro senário onde é importante que o software seja livre, é para proteger os usuários de acesso externo indesejado. Um exemplo disso é o caso do ex analista da NSA \cite{Tate2013}, Edward Joseph Snowden, que em junho de 2013, revelou como a agência americana utilizava de falhas de seguranças, propositais ou não, para espiar na população mundial. 

O principio que o software livre protege a liberdade Individual, contra empresas mau intencionadas ou governos abusivos, é que quando o código de um programa é aberto, a comunidade pode ver oque ele faz e testar todas as falhas que o mesmo possa ter.

\subsection{Software Livre à Favor do Avanço da Computação}
Após a construção dos sistemas operacionais livres, surgiram varias distribuições e variações, cada qual para atender um nicho. Esses avanços, tornaram possível que hoje, os sistemas operacionais baseados no Linux e UNIX dominassem o setor de infra estrutura computacional. Possibilitando que empresas e órgãos governamentais customizassem esses softwares para criar soluções específicas para suas necessidades, que por sua vez não são necessariamente livres.

Essa abordagem se tornou tão prática que, dados da \textit{W3Cook} e \textit{TOP500}, mostram que esses sistemas operacionais são utilizados em $98.3\%$ de todos os servidores públicos de interne e $99.88\%$ dos supercomputadores do mundo. 

A ultima grande plataforma que popularizou a utilização do Linux foi o sistema operacional Android.  Construído inicialmente para ser um software de telefones celulares, esse sistema operacional criado com base no \textit{kernel} do Linux, se espalhou pelos mais diversos aparelhos, como televisões, \textit{tablets} e até mesmo geladeiras. Esse software se popularizou tanto que, segundo o CEO da Google, Sundar Pichai, é utilizado em mais de 2 bilhões de dispositivos ativos.

Além dos avanços em sistemas operacionais, o movimento de software livre alavancou o desenvolvimento comunitário de softwares de computação livres. Um exemplo disso é a \textit{Apache Software Foundation}, que é uma corporação americana sem fins lucrativos formada por uma comunidade descentralizada de desenvolvedores de código aberto. 

Os projetos Apache são feitos em desenvolvimento colaborativo, baseado em consenso e uma licença de software aberta e pragmática. Cada projeto é gerenciado por uma equipe de especialistas técnicos auto-selecionados que são contribuidores ativos para o projeto. 

O projeto inicial da Apache foi o \textit{HTTP Server}, que era um sistema para processar protocolos web básicos na internet. Contudo, hoje são 315 projetos nas mais diversas áreas da computação, e incluem softwares de \textit{Big Data} como o \textit{Spark} e \textit{Hadoop}, gerenciamento de projetos como o \textit{Maven} e até mesmo software de escritório como o \textit{Open Office}.


\subsection{Software Livre à Favor da Acessibilidade da Educação e Conhecimento}

As escolas e universidades, influenciam o futuro da sociedade através do que ensinam. Ensinar um programa proprietário é implantar a dependência de um artifício que não é de sua propriedade. Isso diminui a capacidade do futuro profissional em exercer os conhecimentos que lhe foram ensinados. 

A utilização de software livre como ferramente de ensino, empodera os estudantes a utilizarem os conhecimentos técnicos na vida após a universidade. Utilizando software livre, um profissional é capaz de fazer uso de programas como ferramenta básica de trabalho gratuitamente, ou ainda utilizar soluções livres como base para criar um produto próprio.

Além de ser útil para o estudante, os softwares livres são importantes para o avanço da pesquisa acadêmica na universidade. A comunidade de Software livre têm em suas liberdades básicas, a liberdade de estudar como um programa funciona e permitir que os usuários melhorem e redistribuam esse programa. Este espírito de comunidade permite que pesquisadores em locais diferentes do mundo, sem quaisquer dificuldades, compartilharem suas pesquisas e conhecimentos. 

Além de promover o compartilhamento da informação, os softwares livres também promovem uma  acessibilidade universal dos avanços técnico-científicos, uma vez que permitem pesquisadores e empresas de ponta à compartilhar seus resultados e códigos com o resto do mundo. Isso permite que mesmo pesquisadores com limitantes de recursos, façam aplicações ou pesquisa com esses recursos.

Um exemplo disso, é o caso do software \textit{Tensor Flow} \cite{AbadiABBCCCDDDG16}, feito pela Google. Esse é um programa extremamente complexo que funciona como motor de operações numéricas, utilizando a abstração de computação em grafos de forma escalável para CPU's e GPU's. Além de compartilhar o código do \textit{Tensor Flow} com a comunidade, a empresa também disponibilizou inúmeros modelos de RNA, como a \textit{LeNet} \cite{szegedy2015going} que é um modelo treinado para identificar objetos em imagens. Utilizando esse avanço, pesquisadores do mundo inteiro foram capazes de utilizar esses modelos em suas próprias pesquisas.








