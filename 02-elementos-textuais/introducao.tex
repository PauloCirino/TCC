% -----------------------------------------------------------------------------
%   Arquivo: ./02-elementos-textuais/introducao.tex
% -----------------------------------------------------------------------------

\chapter{Introdução}
\label{chap:introducao}

\section{Justificativa}
\label{sec:justificativa}
Um dos principais objetivos de softwares é auxiliar os usuários nos trabalhos com sistemas computacionais. Uma das formas de atingir esse objetivo é automatizando tarefas rotineiras, repetitivas, desinteressantes e que consomem muito tempo, liberando as pessoas para tarefas mais importantes \cite{Zambiasi2012UmaAA}. 

Uma parte fundamental no ciclo do desenvolvimento de novos algorítimos em aprendizado de máquina é o processo de \textit{benchmarking}. Esse procedimento consiste na comparação do novo método com outros modelos, além, é claro, de testes estatísticos que permitam extrair conclusões quantitativas e qualitativas.

Realizar \textit{benchmarking} pode se tornar uma tarefa repetitiva, trabalhosa e consequentemente cara. O projetista precisa implementar todos os métodos padrões, baixar e pré-processar todas as bases de testes, definir o planejamento e análise dos experimentos, e por fim, realizar testes estatísticos e visualizações de dados para que seja possível extrair suas conclusões.

Dessa forma, a automatização da tarefa de \textit{benchmarking} seria benéfica ao projetista no sentido de economizar tempo para realizar tarefas mais importantes. Além disso, uma rotina de teste padronizada é interessante pois faz com que seja possível a comparação de estudos feitos separadamente.


\section{Objetivo}
\label{sec:objetivo}
Esse trabalho tem como objetivo a criação de um software, que seja capaz de automatizar a tarefa de \textit{benchmarking} para o teste de novos modelos de aprendizado de máquina. 

O projeto será executado na forma de pacote aberto, licença GNU GPLv3, da linguagem de programação estatística R. O pacote será implementado utilizando a própria linguagem R, sua API para $C$ e $C++$ e outros pacotes livres feitos pela comunidade.

Ao fim do ciclo de desenvolvimento, o pacote deverá ser capaz de comparar modelos para tarefas de classificação e regressão desenvolvidos pelos usuários com outros modelos já presentes no pacote.  O usuário poderá fornecer as bases de dados de testes ou utilizar as bases do pacote. O pacote deverá fornecer também a opção de testes estatísticos e visualizações dos resultados.

O objetivo é também ajudar a comunidade com o projeto, dessa forma ele foi enviado ao repositório CRAN e será submetido ao \textit{The R Journal}, um periódico para divulgação de pacotes feitos pela comunidade R.


\section{Organização do trabalho}
\label{sec:organizacaoTrabalho}

O trabalho está estruturado de forma que o capítulo \ref{chap:SoftwareLivre} tem como objetivo contextualizar e discutir os aspectos sociais e culturais do software livre, bem como falar brevemente da história desse movimento e das suas definições de liberdades sociais.

O Capítulo \ref{chap:RProgrammingLanguage}, discutirá sobre a linguagem de programação R, como ela está inserida dentro do movimento de software livre e como ela está organizada. 

Os capítulos \ref{chap:MachineLearning} e \ref{chap:testStatistics} discutirão, de forma sucinta, os conceitos de aprendizado de máquina e de estatísticas de testes que são as bases da implementação do software. 

Por fim, o capítulo \ref{chap:TheSoftware} abordará os casos de uso do projeto de software.

Nesse trabalho não serão discutidas as bases de dados disponíveis no pacote, e tampouco as funções implementadas nele. Esses assuntos são tratadas na documentação do software que se encontra no apêndice \ref{chap:apendiceManual}.
