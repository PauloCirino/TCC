% -----------------------------------------------------------------------------
%   Arquivo: ./02-elementos-textuais/introducao.tex
% -----------------------------------------------------------------------------

\chapter{Introdução}
\label{chap:introducao}

\section{Justificativa}
\label{sec:justificativa}
Um dos principais objetivos de softwares é auxiliar os usuários nos trabalhos com sistemas computacionais. Uma das formas de atingir esse objetivo é automatizando tarefas rotineiras, repetitivas, desinteressantes e que consomem muito tempo. Liberando as pessoas para tarefas mais importantes. 

Uma parte fundamental no ciclo do desenvolvimento de novos algorítimos em aprendizado de máquina é o processo de \textit{benchmarking}. Esse procedimento consiste na comparação do novo método com outros modelos, além é claro de testes estatísticos que permitam extrair conclusões quantitativas e qualitativas.

Realizar \textit{benchmarking} pode se tornar uma tarefa repetitiva e trabalhosa, e consequentemente cara. O engenheiro precisa implementar todos os métodos padrões, ele têm de baixar e pré-processar todas as bases de testes, definir o planejamento e análise dos experimentos, e por fim realizar testes estatísticos e desenho de gráficos para que seja possível extrair suas conclusões.

Dessa forma, a automatização da tarefa de \textit{benchmarking} seria benéfica ao engenheiro, no sentido de economizar tempo para realizar tarefas mais importantes. Além disso, uma rotina de teste padronizada é interessante pois permite que seja possível comparar estudos feitos separadamente.


\section{Objetivo}
\label{sec:objetivo}
Esse trabalho têm como objetivo a criação de um software, que seja capaz de automatizar a tarefa de \textit{benchmarking} para modelos de aprendizado de máquina. 

O projeto será executado na forma de pacote aberto, licença GNU GPLv3, da linguagem de programação estatística R. O pacote ser/a implementado utilizando a própria linguagem R, sua API para $C$ e $C++$ e outros pacotes livres feitos pela comunidade.

Ao fim do ciclo de desenvolvimento, o pacote deverá ser capaz de comparar modelos para tarefas de classificação, clusterização e regressão desenvolvidos pelos usuários com diversos modelos já presentes no pacote.  O usuário poderá fornecer as bases de dados de testes, ou utilizar as bases do pacote. O pacote deverá fornecer também a opção de testes estatísticos e visualizações dos resultados.

Ao fim do desenvolvimento, o software será enviado para publicação no repositório CRAN e será submetido ao \textit{The R Journal}, um periódico para divulgação de pacotes feitos pela comunidade R.


\section{Organização do trabalho}
\label{sec:organizacaoTrabalho}

